\chapter*{Lời mở đầu}

\section*{Vấn đề}
	 Từ những năm gần đây, tình trạng ùn tắc giao thông ở Tp.Hồ Chí Minh đã không dừng lại ở diễn biến phức tạp mà lại còn gia tăng hơn trước. Cụ thể, trên các tuyến đường hiện nay, tình trạng kẹt xe không chỉ xảy ra ở giờ cao điểm mà còn ở các khung giờ khác. Nguyên nhân dẫn đến sự việc trên, một phần ảnh hưởng bởi điều kiện thời tiết, một phần do các công trình cải tạo hạ tầng, khi thi công lấn chiếm mặt đường. Nhưng phần lớn là do mật dộ xe cộ ngày một đông dần, dẫn đến việc ùn ứ, ùn tắt, di chuyển chậm,... Từ những nguyên nhân đó, nhóm sinh viên chúng em đã xây dựng mô hình hệ thống phát hiện kẹt xe từ camera hành trình (cụ thể là camera hành trình xe buýt), với mong muốn có thể góp phần giải quyết được một phần nhỏ tình trạng giao thông hiện nay ở Tp. Hồ Chí Minh nói riêng cũng như ở Việt Nam nói chung.\par
\section*{Động lực nghiên cứu, tính cấp thiết của đề tài}
	Trong bối cảnh mà các công nghệ xử lý trở phát triển mạnh như hiện nay. Chúng ta có thể áp dụng các công nghệ học máy và deep learning trong nhiều lĩnh vực khác nhau từ y tế, nông nghiệp, kinh tế tài chính... Kèm theo đó là lượng dữ liệu tăng đột biến. Sẽ thật tốt nếu có một hệ thống vừa có thể lưu trữ dữ liệu lớn mà vừa có khả năng áp dụng công nghệ học máy, học sâu nói trên để hỗ trợ trong việc xử lý các vấn đề ùn tắt giao thông.\par 
	Khái niệm dữ liệu lớn, học máy, học sâu ở nhiều năm gần đây, khi chính công ty lớn điển hình là Google, hay Tesla cũng đang phát triển tạo ra các hệ thống hỗ trợ giao thông thông minh. Các nước phát triển với các đặc tính dân số đông, mật độ giao thông phức tạp như Trung Quốc cũng đã ráo riết xây dựng các hệ thống camera giám sát kết hớp với học sâu để tiến hành nhận diện ảnh giao thông..v.v.\par 
	Đối với trong nước, hiện nay nhà nước cũng đã bắt đầu tiến hành đầu tư các hạng mục giao thông, thực hiện việc trang bị các camera giám sát tại một số tuyến đường trọng tâm nhằm để theo dõi tinh trạng giao thông. Nhưng để áp dụng lĩnh vực học sâu hay thậm chí là xử lý ảnh vào để sử dụng thì vẫn còn khan hiếm.\par 
	Như vậy, một hệ thống có thể lưu trữ kết hợp với lĩnh vực học sâu là thật sự cần thiết. Không phải chỉ cần thiết cho lĩnh vực giao thông, đây có thể cân nhắc để trở thành một framework, ứng dụng để kết hợp các lĩnh vực khác nhau trong cuộc sống, đây xứng đáng là một đề tài để nghiên cứu và phát triển. Với sự ra đời các mạng học sâu và các framework lưu trữ dữ liệu lớn như Apache Hadoop sẽ hỗ trợ cho việc nghiên cứu và phát triển hệ thống một cách dễ tiếp cận hơn. Chúng em đã nghiên cứu và lựa chọn ra mô hình mạng thích hợp cũng như xây dựng hệ thống mô phỏng lưu trữ dữ liệu lớn mà cho rằng có khả năng áp dụng vào thực tiễn.
	
\section*{Mục tiêu luận văn}		
Với các cơ sở thực tiễn trên, luận văn này đặt ra các mục tiêu như sau:
\begin{itemize}
	\item Lựa chọn, áp dụng mô hình mạng phân loại ảnh giao thông.
	\item Xây dựng hệ thống mô phỏng lưu trữ dữ liệu lớn.
	\item Kết hợp mạng học sâu với hệ thống lưu trữ dữ liệu lớn.
\end{itemize}

\section*{Giới hạn đề tài}
\begin{itemize}
	\item Hướng tới xây dựng mô hình hệ thống phân loại và lưu trữ dữ liệu lớn.
	\item Áp dụng các kiến trúc mạng mã nguôn mở.
	\item Do hướng tới xây dựng hệ thống và áp dụng các kết quả từ các công trình đã có nên luận văn sẽ không chú trọng vào việc cải thiện thuật toán hay đi sâu vào các kiến thức trong lĩnh vực xử lý ảnh.
\end{itemize}

\section*{Tính khả quan}
	Để đánh giá về tính khả quan trong đề tài này, chúng ta cần phân tích một số đặc điểm. Thứ nhất là về tập ảnh giao thông, tập ảnh giao thông tốt hay không tuỳ thuộc vào nhiều yếu tố như nguồn gốc tập ảnh hay cơ cấu hạ tầng camera giao thông được đầu tư như thế nào. Đối với thành phố Hồ Chí Minh nói riêng, chúng ta vẫn còn đang xây dựng hệ thống cơ sở hạ tầng giám sát nên chưa thể dựa vào nguồn camera giám sát để lấy dữ liệu. Vì thế, chúng em sẽ sử dụng tập ảnh từ camera hành trình trên các xe buýt để thực hiện xây dựng mô hình. Khi đã có tập dữ liệu đủ tốt, chúng ta có thể sử dụng mô hình được xây dựng trên đề tài này và có thể được áp dụng vào hệ thống thực tiễn.

\section*{Cấu trúc luận văn}
\begin{itemize}
	\item \textbf{Chương 1: Tổng quan về hệ thống phân loại ảnh giao thống kết hợp với lưu trữ dữ liệu lớn.} Giới thiệu sơ lược về vấn đề đặt ra trong luận văn, các nghiên cứu đã có trên thế giới và hướng tập trung nghiên cứu.
	\item \textbf{Chương 2: Mạng neural và Mạng neural tích chập (Convolutional Neural Networks CNN).} Cơ sở lý thuyết về mạng học sâu.
	\item \textbf{Chương 3: Apache Hadoop.} Tổng quát framework lưu trữ dữ liệu lớn Apache Hadoop.
	\item \textbf{Chương 4: Phương pháp giải quyết.} Đề xuất giải pháp cho đề tài.
	\item \textbf{Chương 5: Hiện thực mô hình loại ảnh giao thông kết hợp Apache Hadoop. }Hiện thực giải pháp đã chọn cho hệ thống phân loại. Bao gồm nội dung về cài đặt và huấn luyện mạng học sâu đã chọn.
	\item \textbf{Chương 6: Kết quả huấn luyện và đánh giá.} Kết quả và các đánh giá về kết quả đạt được.
	\item \textbf{Chương 7: Kết luận.}
\end{itemize}