\chapter{Giới thiệu}

\section{Tổng quan}
	Với sự phát triển nhanh chóng của nước ta, với cơ sở hạ tầng giao thông còn chưa hoàn thiện nên nạn ùn tắt giao thông là điều không thể tránh khỏi. Các công tác nhận biết và xử lý các chốt ùn tắt tương đối chậm và chưa hiệu quả. Nạn ùn tắt giao thông có thể là nguyên nhân chính trong việc kiềm hãm sự phát triển.\par
	Trong những năm gần đây, sự bùng nổ kỹ thuật số và thông tin cho thấy những công trình nghiên cứu mang tính ứng dụng cao trong các lĩnh vực trí tuệ nhân tạo. Điển hình là chủ đề học máy, học sâu, sự phát triển mạnh của hai chủ đề này khiến cho vấn đề phân loại hình ảnh đạt tới một cột mốc mới. Đề tài "Phát hiện kẹt xe qua camera hành trình" kỳ vọng có thể áp ựng mô hình học sâu để nhận biết tình trạng giao thông.
\section{Mục tiêu đề tài}
Đối với đề tài này, giai đoạn đề cương sẽ tập trung tìm hiểu các kiến thưc về học sâu, mạng neuron và các khái niệm cơ bản, mong muốn xây dựng được mô hình có khả năng phân loại hình ảnh giao thông.\par
Các công việc thực hiện:
\begin{itemize}
	\item Tìm hiểu các kiến thức về học sâu nói chung, mạng neuron nói riêng và ứng dụng.
	\item Xây dựng được bộ dữ liệu phục vụ cho việc huấn luyện.
	\item Cấu hình, sử dụng kiến trúc mạng có sẵn để huấn luyện.	
\end{itemize}
