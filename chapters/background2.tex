\chapter{Apache Hadoop}
\section{Tổng quan về hệ thống Apache Hadoop}
Apache Hadoop là một dự án phát triển phần mềm nhằm cung cấp một nền tảng phân tán, có thể mở rộng linh hoạt và có độ tin cậy cao.\\
Ngoài ra Hadoop còn được xem là một thư viện hay framework cho phép xử lý phân tán khối lượng lớn dữ liệu trên nhiều cụm máy tính bằng các mô hình lập trình. Framework nay được thiết kế với mục đích có khả năng mở rộng từ một máy chủ đơn lẻ lên đến rất nhiều trạm làm việc mà mỗi máy trạm có khả năng tính toán và lưu trữ cục bộ.
\section{Hadoop distributed file system - HDFS}
\subsection{Thiết kế}
HDFS là một hệ thống file nhằm lưu trữ một lượng rất rất lớn (Lớn ở đây theo nghĩa có thể là hàng trăm megabytes, gigabytes, hoặc terabytes) với cơ chế streamming data access trên những thiết bị phổ thông \footnote{các máy tính hoặc máy trạm}.
 Đây là cơ chế phân luồng dữ liệu trong HDFS với mục đích ghi một lần chạy nhiều lần. Điển hình là dữ liệu sẽ được sinh và sao chép từ nguồn và sau đó có rất nhiều tiến trình phân tích khác nhau thực thi dữ liệu trên. Mỗi hoạt động phân tích sẽ thực thi liên quan đến một phần nào đó khác nhau trên cả một tập dư liệu trên.\\
 Từ các thiết bị phổ thông ở đây là những những thiệt bị phân cứng máy tính, HDFS không yêu cầu một phần cứng đắt tiền hay độ tin cậy cao. Mà nó được thiết kế để chạy trên những cụm máy tính từ nhiều nhà cung ứng khác nhau. Vì thế mà xác suất để một node(đơn vị phần cứng) gặp lôi và thất bại là rất lớn, đặc biệt là những cụm có hàng ngàn máy trạm. Với đặc tính đó, HDFS được thiết kế sao cho không có sự gián đoạn nào được phát hiện ở người dùng khi mà việc một số lượng node gặp lỗi giữa chừng.
\subsection{Ý tưởng chủ đạo}
